%%%%%%%%%%%%%%%%%%%%%%%%%%%%%%%%%%%%%%%%%%%%%%%%%%%%%%%%%%%%%%%%%%%%%%%%
%%%%%%%%%%%%%%%%%%%%%% Simple LaTeX CV Template %%%%%%%%%%%%%%%%%%%%%%%%
%%%%%%%%%%%%%%%%%%%%%%%%%%%%%%%%%%%%%%%%%%%%%%%%%%%%%%%%%%%%%%%%%%%%%%%%

%%%%%%%%%%%%%%%%%%%%%%%%%%%%%%%%%%%%%%%%%%%%%%%%%%%%%%%%%%%%%%%%%%%%%%%%
%% NOTE: If you find that it says                                     %%
%%                                                                    %%
%%                           1 of ??                                  %%
%%                                                                    %%
%% at the bottom of your first page, this means that the AUX file     %%
%% was not available when you ran LaTeX on this source. Simply RERUN  %%
%% LaTeX to get the ``??'' replaced with the number of the last page  %%
%% of the document. The AUX file will be generated on the first run   %%
%% of LaTeX and used on the second run to fill in all of the          %%
%% references.                                                        %%
%%%%%%%%%%%%%%%%%%%%%%%%%%%%%%%%%%%%%%%%%%%%%%%%%%%%%%%%%%%%%%%%%%%%%%%%

%%%%%%%%%%%%%%%%%%%%%%%%%%%% Document Setup %%%%%%%%%%%%%%%%%%%%%%%%%%%%

% Don't like 10pt? Try 11pt or 12pt
\documentclass[10pt]{article}

% This is a helpful package that puts math inside length specifications
\usepackage{calc}
\usepackage[utf8]{inputenc}

% Simpler bibsection for CV sections
% (thanks to natbib for inspiration)
\makeatletter
\newlength{\bibhang}
\setlength{\bibhang}{1em}
\newlength{\bibsep}
{\@listi \global\bibsep\itemsep \global\advance\bibsep by\parsep}
\newenvironment{bibsection}%
{\begin{list}{}{%
  \setlength{\leftmargin}{\bibhang}%
  \setlength{\itemindent}{-\leftmargin}%
  \setlength{\itemsep}{\bibsep}%
  \setlength{\parsep}{\z@}%
  \setlength{\partopsep}{0pt}%
  \setlength{\topsep}{0pt}}}
  {\end{list}\vspace{-.6\baselineskip}}
  \makeatother

% Layout: Puts the section titles on left side of page
\reversemarginpar

%
%         PAPER SIZE, PAGE NUMBER, AND DOCUMENT LAYOUT NOTES:
%
% The next \usepackage line changes the layout for CV style section
% headings as marginal notes. It also sets up the paper size as either
% letter or A4. By default, letter was used. If A4 paper is desired,
% comment out the letterpaper lines and uncomment the a4paper lines.
%
% As you can see, the margin widths and section title widths can be
% easily adjusted.
%
% ALSO: Notice that the includefoot option can be commented OUT in order
% to put the PAGE NUMBER *IN* the bottom margin. This will make the
% effective text area larger.
%
% IF YOU WISH TO REMOVE THE ``of LASTPAGE'' next to each page number,
% see the note about the +LP and -LP lines below. Comment out the +LP
% and uncomment the -LP.
%
% IF YOU WISH TO REMOVE PAGE NUMBERS, be sure that the includefoot line
% is uncommented and ALSO uncomment the \pagestyle{empty} a few lines
% below.
%

%% Use these lines for letter-sized paper
%\usepackage[paper=letterpaper,
%            %includefoot, % Uncomment to put page number above margin
%            marginparwidth=1.2in,     % Length of section titles
%            marginparsep=.05in,       % Space between titles and text
%            margin=1in,               % 1 inch margins
%            includemp]{geometry}

%% Use these lines for A4-sized paper
\usepackage[paper=a4paper,
            %includefoot, % Uncomment to put page number above margin
            marginparwidth=30.5mm,    % Length of section titles
            marginparsep=1.5mm,       % Space between titles and text
            margin=20mm,              % 25mm margins
            includemp]{geometry}

%% More layout: Get rid of indenting throughout entire document
\setlength{\parindent}{0in}

\usepackage[shortlabels]{enumitem}

%% Reference the last page in the page number
%
% NOTE: comment the +LP line and uncomment the -LP line to have page
%       numbers without the ``of ##'' last page reference)
%
% NOTE: uncomment the \pagestyle{empty} line to get rid of all page
%       numbers (make sure includefoot is commented out above)
%
\usepackage{fancyhdr,lastpage}
\pagestyle{fancy}
\pagestyle{empty}      % Uncomment this to get rid of page numbers
\fancyhf{}\renewcommand{\headrulewidth}{0pt}
\fancyfootoffset{\marginparsep+\marginparwidth}
\newlength{\footpageshift}
\setlength{\footpageshift}
          {0.5\textwidth+0.5\marginparsep+0.5\marginparwidth-2in}
\lfoot{\hspace{\footpageshift}%
       \parbox{4in}{\, \hfill %
                    \arabic{page} of \protect\pageref*{LastPage} % +LP
%                    \arabic{page}                               % -LP
                    \hfill \,}}

% Finally, give us PDF bookmarks
\usepackage{color,hyperref}
\definecolor{darkblue}{rgb}{0.0,0.0,0.3}
\hypersetup{colorlinks,breaklinks,
            linkcolor=darkblue,urlcolor=darkblue,
            anchorcolor=darkblue,citecolor=darkblue}

%%%%%%%%%%%%%%%%%%%%%%%% End Document Setup %%%%%%%%%%%%%%%%%%%%%%%%%%%%


%%%%%%%%%%%%%%%%%%%%%%%%%%% Helper Commands %%%%%%%%%%%%%%%%%%%%%%%%%%%%

% The title (name) with a horizontal rule under it
% (optional argument typesets an object right-justified across from name
%  as well)
%
% Usage: \makeheading{name}
%        OR
%        \makeheading[right_object]{name}
%
% Place at top of document. It should be the first thing.
% If ``right_object'' is provided in the square-braced optional
% argument, it will be right justified on the same line as ``name'' at
% the top of the CV. For example:
%
%       \makeheading[\emph{Curriculum vitae}]{Your Name}
%
% will put an emphasized ``Curriculum vitae'' at the top of the document
% as a title. Likewise, a picture could be included:
%
%   \makeheading[\includegraphics[height=1.5in]{my_picutre}]{Your Name}
%
% the picture will be flush right across from the name.
\newcommand{\makeheading}[2][]%
        {\hspace*{-\marginparsep minus \marginparwidth}%
         \begin{minipage}[t]{\textwidth+\marginparwidth+\marginparsep}%
             {\large \bfseries #2 \hfill #1}\\[-0.15\baselineskip]%
                 \rule{\columnwidth}{1pt}%
         \end{minipage}}

% The section headings
%
% Usage: \section{section name}
\renewcommand{\section}[1]{\pagebreak[3]%
    \hyphenpenalty=10000%
    \vspace{1.3\baselineskip}%
    \phantomsection\addcontentsline{toc}{section}{#1}%
    \noindent\llap{\scshape\smash{\parbox[t]{\marginparwidth}{\raggedright #1}}}%
    \vspace{-\baselineskip}\par}

% An itemize-style list with lots of space between items
\newenvironment{outerlist}[1][\enskip\textbullet]%
        {\begin{itemize}[#1,leftmargin=*]}{\end{itemize}%
         \vspace{-.6\baselineskip}}

% An environment IDENTICAL to outerlist that has better pre-list spacing
% when used as the first thing in a \section
\newenvironment{lonelist}[1][\enskip\textbullet]%
        {\begin{list}{#1}{%
        \setlength{\partopsep}{0pt}%
        \setlength{\topsep}{0pt}}}
        {\end{list}\vspace{-.6\baselineskip}}

% An itemize-style list with little space between items
\newenvironment{innerlist}[1][\enskip\textbullet]%
        {\begin{itemize}[#1,leftmargin=*,parsep=0pt,itemsep=0pt,topsep=0pt,partopsep=0pt]}
        {\end{itemize}}

% An environment IDENTICAL to innerlist that has better pre-list spacing
% when used as the first thing in a \section
\newenvironment{loneinnerlist}[1][\enskip\textbullet]%
        {\begin{itemize}[#1,leftmargin=*,parsep=0pt,itemsep=0pt,topsep=0pt,partopsep=0pt]}
        {\end{itemize}\vspace{-.6\baselineskip}}

% To add some paragraph space between lines.
% This also tells LaTeX to preferably break a page on one of these gaps
% if there is a needed pagebreak nearby.
\newcommand{\blankline}{\quad\pagebreak[3]}
\newcommand{\halfblankline}{\quad\vspace{-0.5\baselineskip}\pagebreak[3]}

% Uses hyperref to link DOI
\newcommand\doilink[1]{\href{http://dx.doi.org/#1}{#1}}
\newcommand\doi[1]{doi:\doilink{#1}}

% For \url{SOME_URL}, links SOME_URL to the url SOME_URL
\providecommand*\url[1]{\href{#1}{#1}}
% Same as above, but pretty-prints SOME_URL in teletype fixed-width font
\renewcommand*\url[1]{\href{#1}{\texttt{#1}}}

% For \email{ADDRESS}, links ADDRESS to the url mailto:ADDRESS
\providecommand*\email[1]{\href{mailto:#1}{#1}}
% Same as above, but pretty-prints ADDRESS in teletype fixed-width font
%\renewcommand*\email[1]{\href{mailto:#1}{\texttt{#1}}}

%\providecommand\BibTeX{{\rm B\kern-.05em{\sc i\kern-.025em b}\kern-.08em
%    T\kern-.1667em\lower.7ex\hbox{E}\kern-.125emX}}
%\providecommand\BibTeX{{\rm B\kern-.05em{\sc i\kern-.025em b}\kern-.08em
%    \TeX}}
\providecommand\BibTeX{{B\kern-.05em{\sc i\kern-.025em b}\kern-.08em
    \TeX}}
\providecommand\Matlab{\textsc{Matlab}}

%%%%%%%%%%%%%%%%%%%%%%%% End Helper Commands %%%%%%%%%%%%%%%%%%%%%%%%%%%

%%%%%%%%%%%%%%%%%%%%%%%%% Begin CV Document %%%%%%%%%%%%%%%%%%%%%%%%%%%%

\begin{document}
\makeheading{Ian L. Rodrigues -- 28 years}

\section{Contact}

% NOTE: Mind where the & separators and \\ breaks are in the following
%       table.
%
% ALSO: \rcollength is the width of the right column of the table
%       (adjust it to your liking; default is 1.85in).
%
\newlength{\rcollength}\setlength{\rcollength}{1.85in}%
%
\begin{tabular}[t]{@{}p{\textwidth-\rcollength}p{\rcollength}}
\textit{Phone:} +55 19 99899-5021          \\
\textit{E-mail:} \email{ian.liu88@gmail.com}
\end{tabular}

\section{Languages}

\textbf{Portuguese} -- Native language

\textbf{English} -- Advanced

\section{Education}

Coursera
\begin{outerlist}
\item[] Introduction to Machine Learning.\hfill {\bf 2014}
\end{outerlist}

University of Campinas
\begin{outerlist}
\item[] BSc in Computational and Applied Mathmatics.\hfill {\bf 2006 -- 2010}
\end{outerlist}

\section{Experiences}

\begin{outerlist}

\item[] Main Software Developer at CEPETRO\hfill {\bf 09/2012 -- current}
  \begin{innerlist}
  \item Develop highly optimized algorithms, heuristics for function
maximization, various interpolation techniques---all applied to geophysics
problems at the Center for Petroleum Studies (cepetro.com.br). Working mainly
with C++ and Python.
  
  \end{innerlist}

\item[] Junior Analyst at Itaú Bank\hfill {\bf 04/2012 -- 08/2012}
  \begin{innerlist}
  \item Developed database systems and web services using the .NET platform.
  \end{innerlist}

\item[] Software Developer at GêBR\hfill {\bf 08/2009 -- 2013}
  \begin{innerlist}
  \item The GêBR project (gebrproject.com) is a Graphical User
  Interface tool for geophysicists and geologists to process seismic data. It
  was financed by Petrobras as a research project.
  \end{innerlist}

\item[] Open-Source Contribution to Ubuntu\hfill {\bf 03/2011}
  \begin{innerlist}
  \item Fixed the following two bugs using the Vala language
  \item \href{https://bugs.launchpad.net/unity-lens-applications/+bug/734762}{bugs.launchpad.net/unity-lens-applications/+bug/734762}
  \item \href{https://bugs.launchpad.net/unity-lens-applications/+bug/736471}{bugs.launchpad.net/unity-lens-applications/+bug/736471}
  \end{innerlist}

\item[] Freelance job for {\sc Skedio Tec}\hfill {\bf 02/2009}
  \begin{innerlist}
  \item Program in Python that plots real-time data collected from a probe that
  measures moisture in concrete tanks.
  \end{innerlist}

\item[] Main Software Developer at IgnisCom\hfill {\bf 04/2007 -- 06/2008}
  \begin{innerlist}
  \item Developed educational games using Flash/ActionScript platform.
  \end{innerlist}

\item[] Educational Flash Applets\hfill {\bf 2006}
  \begin{innerlist}
  \item Developed programs in Flash to exhibit curiosities of geometry for
  Prof. Alberto Saa, at Unicamp (github.com/ianliu/curiosidades-da-geometria)
  \end{innerlist}

\end{outerlist}

\section{Events}

\begin{outerlist}

\item[] Intel Tech Theater\hfill {\bf 2015}
  \begin{innerlist}
  \item Presented the ``Efficient and Fault Tolerant Computation of Partially
  Idempotent Tasks'' expanded abstract at the Intel Tech Theater, Rio de Janeiro.
  \end{innerlist}

\item[] SBGf 2015\hfill {\bf 2015}
  \begin{innerlist}
  \item Published the ``Efficient and Fault Tolerant Computation of Partially
  Idempotent Tasks'' expanded abstract at Geophysics Brazilian Society
  conference, at Rio de Janeiro.
  \end{innerlist}

\item[] Wave Inversion Technology\hfill {\bf 2013}
  \begin{innerlist}
  \item Presented the report ``Enabling large data processing with the 3D ZO CRS
  Stack software'' at Hamburgo, Germany.
  \end{innerlist}

\item[] ERAD Contest\hfill {\bf 2013}
  \begin{innerlist}
  \item Participated at the High Performance Computing Contest at the ERAD
  congress, São Paulo. Developed a very efficient Image Blur algorithm using
  SIMD instructions (github.com/ianliu/desafio-erad-sp-2013).
  \end{innerlist}

\item[] 12th CISBGf\hfill {\bf 08/2011}
  \begin{innerlist}
  \item Participated at the 12th CISBGf to present the GêBR project, at Rio de Janeiro.
  \end{innerlist}

\item[] Google Developer Day\hfill {\bf 10/2010}
  \begin{innerlist}
  \item Attended the Google Developer Day, at São Paulo.
  \end{innerlist}

\item[] 11th CISGBf\hfill {\bf 08/2009}
  \begin{innerlist}
  \item Participated at the 11th CISBGf to present the GêBR project, at Rio de Janeiro.
  \end{innerlist}

\item[] Summer Course at LNCC\hfill {\bf 01/2009}
  \begin{innerlist}
  \item Attended the summer course at the National Laboratory of Computational
  Science (LNCC), located at Petrópolis, Rio de Janeiro. Classes taken include
  Monte Carlo method, Quantic Computing, and Graphs Algorithms.
  \end{innerlist}

\end{outerlist}

\section{Skills}

\begin{outerlist}

\item[] Math:
  \begin{innerlist}
  \item Interpolation Methods, Optimization Algorithms and Heuristics, Random Walk
  \end{innerlist}

\item[] Computing:
  \begin{innerlist}
  \item C/C++, Python, Shell Scripting, JavaScript, SQL
  \item Linux systems, Version Controls, Trello, GitHub
  \end{innerlist}

\item[] Finance:
  \begin{innerlist}
  \item Value Investing
  \item Stocks and Options modeling
  \end{innerlist}

\item[] Hobbies:
  \begin{innerlist}
  \item Gardening, Kung Fu, Dogs
  \end{innerlist}

\end{outerlist}

\section{Links}

\begin{outerlist}

\item[] Links:
  \begin{innerlist}
  \item github.com/ianliu
  \item facebook.com/IanLiuRodrigues
  \end{innerlist}

\end{outerlist}

\end{document}

%%%%%%%%%%%%%%%%%%%%%%%%%% End CV Document %%%%%%%%%%%%%%%%%%%%%%%%%%%%%

%----------------------------------------------------------------------%
% The following is copyright and licensing information for
% redistribution of this LaTeX source code; it also includes a liability
% statement. If this source code is not being redistributed to others,
% it may be omitted. It has no effect on the function of the above code.
%----------------------------------------------------------------------%
% Copyright (c) 2007, 2008, 2009, 2010, 2011 by Theodore P. Pavlic
%
% Unless otherwise expressly stated, this work is licensed under the
% Creative Commons Attribution-Noncommercial 3.0 United States License. To
% view a copy of this license, visit
% http://creativecommons.org/licenses/by-nc/3.0/us/ or send a letter to
% Creative Commons, 171 Second Street, Suite 300, San Francisco,
% California, 94105, USA.
%
% THE SOFTWARE IS PROVIDED "AS IS", WITHOUT WARRANTY OF ANY KIND, EXPRESS
% OR IMPLIED, INCLUDING BUT NOT LIMITED TO THE WARRANTIES OF
% MERCHANTABILITY, FITNESS FOR A PARTICULAR PURPOSE AND NONINFRINGEMENT.
% IN NO EVENT SHALL THE AUTHORS OR COPYRIGHT HOLDERS BE LIABLE FOR ANY
% CLAIM, DAMAGES OR OTHER LIABILITY, WHETHER IN AN ACTION OF CONTRACT,
% TORT OR OTHERWISE, ARISING FROM, OUT OF OR IN CONNECTION WITH THE
% SOFTWARE OR THE USE OR OTHER DEALINGS IN THE SOFTWARE.
%----------------------------------------------------------------------%
