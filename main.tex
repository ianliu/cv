%%%%%%%%%%%%%%%%%%%%%%%%%%%%%%%%%%%%%%%%%%%%%%%%%%%%%%%%%%%%%%%%%%%%%%%%
%%%%%%%%%%%%%%%%%%%%%% Simple LaTeX CV Template %%%%%%%%%%%%%%%%%%%%%%%%
%%%%%%%%%%%%%%%%%%%%%%%%%%%%%%%%%%%%%%%%%%%%%%%%%%%%%%%%%%%%%%%%%%%%%%%%

%%%%%%%%%%%%%%%%%%%%%%%%%%%%%%%%%%%%%%%%%%%%%%%%%%%%%%%%%%%%%%%%%%%%%%%%
%% NOTE: If you find that it says                                     %%
%%                                                                    %%
%%                           1 of ??                                  %%
%%                                                                    %%
%% at the bottom of your first page, this means that the AUX file     %%
%% was not available when you ran LaTeX on this source. Simply RERUN  %%
%% LaTeX to get the ``??'' replaced with the number of the last page  %%
%% of the document. The AUX file will be generated on the first run   %%
%% of LaTeX and used on the second run to fill in all of the          %%
%% references.                                                        %%
%%%%%%%%%%%%%%%%%%%%%%%%%%%%%%%%%%%%%%%%%%%%%%%%%%%%%%%%%%%%%%%%%%%%%%%%

%%%%%%%%%%%%%%%%%%%%%%%%%%%% Document Setup %%%%%%%%%%%%%%%%%%%%%%%%%%%%

% Don't like 10pt? Try 11pt or 12pt
\documentclass[10pt]{article}

% This is a helpful package that puts math inside length specifications
\usepackage{calc}
\usepackage[utf8]{inputenc}

% Simpler bibsection for CV sections
% (thanks to natbib for inspiration)
\makeatletter
\newlength{\bibhang}
\setlength{\bibhang}{1em}
\newlength{\bibsep}
{\@listi \global\bibsep\itemsep \global\advance\bibsep by\parsep}
\newenvironment{bibsection}%
{\begin{list}{}{%
  \setlength{\leftmargin}{\bibhang}%
  \setlength{\itemindent}{-\leftmargin}%
  \setlength{\itemsep}{\bibsep}%
  \setlength{\parsep}{\z@}%
  \setlength{\partopsep}{0pt}%
  \setlength{\topsep}{0pt}}}
  {\end{list}\vspace{-.6\baselineskip}}
  \makeatother

% Layout: Puts the section titles on left side of page
\reversemarginpar

%
%         PAPER SIZE, PAGE NUMBER, AND DOCUMENT LAYOUT NOTES:
%
% The next \usepackage line changes the layout for CV style section
% headings as marginal notes. It also sets up the paper size as either
% letter or A4. By default, letter was used. If A4 paper is desired,
% comment out the letterpaper lines and uncomment the a4paper lines.
%
% As you can see, the margin widths and section title widths can be
% easily adjusted.
%
% ALSO: Notice that the includefoot option can be commented OUT in order
% to put the PAGE NUMBER *IN* the bottom margin. This will make the
% effective text area larger.
%
% IF YOU WISH TO REMOVE THE ``of LASTPAGE'' next to each page number,
% see the note about the +LP and -LP lines below. Comment out the +LP
% and uncomment the -LP.
%
% IF YOU WISH TO REMOVE PAGE NUMBERS, be sure that the includefoot line
% is uncommented and ALSO uncomment the \pagestyle{empty} a few lines
% below.
%

%% Use these lines for letter-sized paper
%\usepackage[paper=letterpaper,
%            %includefoot, % Uncomment to put page number above margin
%            marginparwidth=1.2in,     % Length of section titles
%            marginparsep=.05in,       % Space between titles and text
%            margin=1in,               % 1 inch margins
%            includemp]{geometry}

%% Use these lines for A4-sized paper
\usepackage[paper=a4paper,
            %includefoot, % Uncomment to put page number above margin
            marginparwidth=30.5mm,    % Length of section titles
            marginparsep=1.5mm,       % Space between titles and text
            margin=20mm,              % 25mm margins
            includemp]{geometry}

%% More layout: Get rid of indenting throughout entire document
\setlength{\parindent}{0in}

\usepackage[shortlabels]{enumitem}

%% Reference the last page in the page number
%
% NOTE: comment the +LP line and uncomment the -LP line to have page
%       numbers without the ``of ##'' last page reference)
%
% NOTE: uncomment the \pagestyle{empty} line to get rid of all page
%       numbers (make sure includefoot is commented out above)
%
\usepackage{fancyhdr,lastpage}
\pagestyle{fancy}
\pagestyle{empty}      % Uncomment this to get rid of page numbers
\fancyhf{}\renewcommand{\headrulewidth}{0pt}
\fancyfootoffset{\marginparsep+\marginparwidth}
\newlength{\footpageshift}
\setlength{\footpageshift}
          {0.5\textwidth+0.5\marginparsep+0.5\marginparwidth-2in}
\lfoot{\hspace{\footpageshift}%
       \parbox{4in}{\, \hfill %
                    \arabic{page} of \protect\pageref*{LastPage} % +LP
%                    \arabic{page}                               % -LP
                    \hfill \,}}

% Finally, give us PDF bookmarks
\usepackage{color,hyperref}
\definecolor{darkblue}{rgb}{0.0,0.0,0.3}
\hypersetup{colorlinks,breaklinks,
            linkcolor=darkblue,urlcolor=darkblue,
            anchorcolor=darkblue,citecolor=darkblue}

%%%%%%%%%%%%%%%%%%%%%%%% End Document Setup %%%%%%%%%%%%%%%%%%%%%%%%%%%%


%%%%%%%%%%%%%%%%%%%%%%%%%%% Helper Commands %%%%%%%%%%%%%%%%%%%%%%%%%%%%

% The title (name) with a horizontal rule under it
% (optional argument typesets an object right-justified across from name
%  as well)
%
% Usage: \makeheading{name}
%        OR
%        \makeheading[right_object]{name}
%
% Place at top of document. It should be the first thing.
% If ``right_object'' is provided in the square-braced optional
% argument, it will be right justified on the same line as ``name'' at
% the top of the CV. For example:
%
%       \makeheading[\emph{Curriculum vitae}]{Your Name}
%
% will put an emphasized ``Curriculum vitae'' at the top of the document
% as a title. Likewise, a picture could be included:
%
%   \makeheading[\includegraphics[height=1.5in]{my_picutre}]{Your Name}
%
% the picture will be flush right across from the name.
\newcommand{\makeheading}[2][]%
        {\hspace*{-\marginparsep minus \marginparwidth}%
         \begin{minipage}[t]{\textwidth+\marginparwidth+\marginparsep}%
             {\large \bfseries #2 \hfill #1}\\[-0.15\baselineskip]%
                 \rule{\columnwidth}{1pt}%
         \end{minipage}}

% The section headings
%
% Usage: \section{section name}
\renewcommand{\section}[1]{\pagebreak[3]%
    \hyphenpenalty=10000%
    \vspace{1.3\baselineskip}%
    \phantomsection\addcontentsline{toc}{section}{#1}%
    \noindent\llap{\scshape\smash{\parbox[t]{\marginparwidth}{\raggedright #1}}}%
    \vspace{-\baselineskip}\par}

% An itemize-style list with lots of space between items
\newenvironment{outerlist}[1][\enskip\textbullet]%
        {\begin{itemize}[#1,leftmargin=*]}{\end{itemize}%
         \vspace{-.6\baselineskip}}

% An environment IDENTICAL to outerlist that has better pre-list spacing
% when used as the first thing in a \section
\newenvironment{lonelist}[1][\enskip\textbullet]%
        {\begin{list}{#1}{%
        \setlength{\partopsep}{0pt}%
        \setlength{\topsep}{0pt}}}
        {\end{list}\vspace{-.6\baselineskip}}

% An itemize-style list with little space between items
\newenvironment{innerlist}[1][\enskip\textbullet]%
        {\begin{itemize}[#1,leftmargin=*,parsep=0pt,itemsep=0pt,topsep=0pt,partopsep=0pt]}
        {\end{itemize}}

% An environment IDENTICAL to innerlist that has better pre-list spacing
% when used as the first thing in a \section
\newenvironment{loneinnerlist}[1][\enskip\textbullet]%
        {\begin{itemize}[#1,leftmargin=*,parsep=0pt,itemsep=0pt,topsep=0pt,partopsep=0pt]}
        {\end{itemize}\vspace{-.6\baselineskip}}

% To add some paragraph space between lines.
% This also tells LaTeX to preferably break a page on one of these gaps
% if there is a needed pagebreak nearby.
\newcommand{\blankline}{\quad\pagebreak[3]}
\newcommand{\halfblankline}{\quad\vspace{-0.5\baselineskip}\pagebreak[3]}

% Uses hyperref to link DOI
\newcommand\doilink[1]{\href{http://dx.doi.org/#1}{#1}}
\newcommand\doi[1]{doi:\doilink{#1}}

% For \url{SOME_URL}, links SOME_URL to the url SOME_URL
\providecommand*\url[1]{\href{#1}{#1}}
% Same as above, but pretty-prints SOME_URL in teletype fixed-width font
\renewcommand*\url[1]{\href{#1}{\texttt{#1}}}

% For \email{ADDRESS}, links ADDRESS to the url mailto:ADDRESS
\providecommand*\email[1]{\href{mailto:#1}{#1}}
% Same as above, but pretty-prints ADDRESS in teletype fixed-width font
%\renewcommand*\email[1]{\href{mailto:#1}{\texttt{#1}}}

%\providecommand\BibTeX{{\rm B\kern-.05em{\sc i\kern-.025em b}\kern-.08em
%    T\kern-.1667em\lower.7ex\hbox{E}\kern-.125emX}}
%\providecommand\BibTeX{{\rm B\kern-.05em{\sc i\kern-.025em b}\kern-.08em
%    \TeX}}
\providecommand\BibTeX{{B\kern-.05em{\sc i\kern-.025em b}\kern-.08em
    \TeX}}
\providecommand\Matlab{\textsc{Matlab}}

%%%%%%%%%%%%%%%%%%%%%%%% End Helper Commands %%%%%%%%%%%%%%%%%%%%%%%%%%%

%%%%%%%%%%%%%%%%%%%%%%%%% Begin CV Document %%%%%%%%%%%%%%%%%%%%%%%%%%%%

\begin{document}
\makeheading{Ian L. Rodrigues -- 23 anos}

\section{Contato}

% NOTE: Mind where the & separators and \\ breaks are in the following
%       table.
%
% ALSO: \rcollength is the width of the right column of the table
%       (adjust it to your liking; default is 1.85in).
%
\newlength{\rcollength}\setlength{\rcollength}{1.85in}%
%
\begin{tabular}[t]{@{}p{\textwidth-\rcollength}p{\rcollength}}
\textit{Celular:} +55-19-9707-8395           & R. José Antônio Poletto, 336  \\
\textit{E-mail:} \email{ian.liu88@gmail.com} & Res. Pq. Rio das Pedras       \\
                                             & CEP 13085-135                 \\
                                             & Campinas, SP                    
\end{tabular}

\section{Objetivo}

Construir uma carreira em uma grande empresa que possibilite crescimento
profissional e pessoal.

\section{Idiomas}

\textbf{Inglês} -- Fala bem; Compreende bem; Escreve bem; Lê bem.

\section{Educação}

Universidade Estadual de Campinas
\begin{outerlist}
\item[] Mestrado em Ciência da Computação.\hfill {\bf 2010 -- Atual}
\item[] Bacharel em Matemática Aplicada e Computacional.\hfill {\bf 2006 -- 2010}
\end{outerlist}

\section{Experiências}

\begin{outerlist}
\item[] Desenvolvedor de Software em C\hfill {\bf 08/2009 -- Atual}
  \begin{innerlist}
  \item Projeto \href{http://www.gebrproject.com}{GêBR}, escrito em C e GTK+.
  \end{innerlist}

\item[] Contribuição ao Ubuntu\hfill {\bf 03/2011}
  \begin{innerlist}
  \item Correção de dois {\em bugs} no Ubuntu utilizando a linguagem Vala.
  \item https://bugs.launchpad.net/unity-lens-applications/+bug/734762
  \item https://bugs.launchpad.net/unity-lens-applications/+bug/736471
  \end{innerlist}

\item[] Projeto GêBR\hfill {\bf 03/2009 -- 07/2009}
  \begin{innerlist}
  \item Contribuições ao projeto livre, GêBR.
  \end{innerlist}

\item[] Trabalho para {\sc Skedio Tec}\hfill {\bf 02/2009}
  \begin{innerlist}
  \item Programa em {\em Python} que plota um gráfico em tempo real dos dados
    de uma sonda através da porta serial RS-232.
  \end{innerlist}

\item[] Desenvolvedor na IgnisCom\hfill {\bf 04/2007 -- 06/2008}
  \begin{innerlist}
  \item Projeto de jogos educativos desenvolvidos em Flash/ActionScript.
  \end{innerlist}

\item[] {\em Applets} em Flash\hfill {\bf 2006}
  \begin{innerlist}
  \item Projeto para o Prof. Alberto Saa da Unicamp: applets em Flash que
    exibiam curiosidades geométricas.
  \end{innerlist}

\end{outerlist}

\section{Eventos}

\begin{outerlist}

\item[] Curso de Android\hfill {\bf 09/2011 -- Atual}
  \begin{innerlist}
  \item Curso de Android oferecido pela Motorola e instituto Eldorado.
  \end{innerlist}

\item[] 12º CISBGf\hfill {\bf 08/2011}
  \begin{innerlist}
  \item Participação no 12º CISBGf para apresentar o projeto GêBR no Rio de Janeiro.
  \end{innerlist}

\item[] 11º CISGBf\hfill {\bf 08/2009}
  \begin{innerlist}
  \item Participação no 11º CISBGf para apresentar o projeto GêBR na Bahia.
  \end{innerlist}

\item[] Curso de verão no LNCC\hfill {\bf 01/2009}
  \begin{innerlist}
  \item Participação no curso de verão no LNCC - Laboratório Nacional de
    Computação Científica em Petrópolis. Aulas vistas incluem método Monte
    Carlo, computação quântica e grafos.
  \end{innerlist}

\end{outerlist}

\section{Conhecimentos}

\begin{outerlist}

\item[] Programação:
  \begin{innerlist}
  \item C/C++, Python, Vala, Autotools, shell scripting, Java, JavaScript, SQL
  \end{innerlist}

\item[] Controles de Versão:
  \begin{innerlist}
  \item Mercurial, Git, SVN, Bazaar
  \end{innerlist}

\end{outerlist}

\end{document}

%%%%%%%%%%%%%%%%%%%%%%%%%% End CV Document %%%%%%%%%%%%%%%%%%%%%%%%%%%%%

%----------------------------------------------------------------------%
% The following is copyright and licensing information for
% redistribution of this LaTeX source code; it also includes a liability
% statement. If this source code is not being redistributed to others,
% it may be omitted. It has no effect on the function of the above code.
%----------------------------------------------------------------------%
% Copyright (c) 2007, 2008, 2009, 2010, 2011 by Theodore P. Pavlic
%
% Unless otherwise expressly stated, this work is licensed under the
% Creative Commons Attribution-Noncommercial 3.0 United States License. To
% view a copy of this license, visit
% http://creativecommons.org/licenses/by-nc/3.0/us/ or send a letter to
% Creative Commons, 171 Second Street, Suite 300, San Francisco,
% California, 94105, USA.
%
% THE SOFTWARE IS PROVIDED "AS IS", WITHOUT WARRANTY OF ANY KIND, EXPRESS
% OR IMPLIED, INCLUDING BUT NOT LIMITED TO THE WARRANTIES OF
% MERCHANTABILITY, FITNESS FOR A PARTICULAR PURPOSE AND NONINFRINGEMENT.
% IN NO EVENT SHALL THE AUTHORS OR COPYRIGHT HOLDERS BE LIABLE FOR ANY
% CLAIM, DAMAGES OR OTHER LIABILITY, WHETHER IN AN ACTION OF CONTRACT,
% TORT OR OTHERWISE, ARISING FROM, OUT OF OR IN CONNECTION WITH THE
% SOFTWARE OR THE USE OR OTHER DEALINGS IN THE SOFTWARE.
%----------------------------------------------------------------------%
